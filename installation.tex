%!TEX root = thesis.tex

\chapter{Installation}
\label{chapter-installation}

Dieser Abschnitt beschreibt die notwendigen Schritte zur Installation von BS4 und Requests.
Eine vorhandene Installation von Python 3 wird vorausgesetzt.

\section{BeautifulSoup (BS4)}

Die Installation von BS4 kann gem. Dokumentation\cite{bs4} bei den Linux Derivaten Debian oder Ubuntu über den integrierten System Paket Manager apt-get durchgeführt werden.

\texttt{\$ apt-get install python3-bs4}

Ist eine Installation über einen System Paket Manager nicht möglich, gibt es die Möglichkeit, BS4 über die Paket Manager Easy Install und PIP zu installieren.

\texttt{\$ easy\_install beautifulsoup4}

\texttt{\$ pip install beautifulsoup4}

Bei älteren Versionen von Python, kann es zu Problemen bei der Installation kommen.
Es ist darauf zu achten, dass die passenden Paket Manager für Version 3 verwendet werden. 
Gegebenenfalls muss die Installation alternativ mit den Kommandos \texttt{easy\_install3} oder \texttt{pip3} durchgeführt werden.

\section{Requests}

Die Installation von \texttt{requests} erfolgt ebenfalls über den Paket Manager \texttt{pip}\cite{pypi-requests}.

\texttt{\$ pip install requests}

In der offiziellen Dokumentation des Pakets wird zusätzlich eine Variante über den Source Code bschrieben, sollte es Probleme geben\cite{requests-readthedocs}.